
%
\documentclass[useAMS,usenatbib,referee]{article}
%
%
% \usepackage{Times}
\usepackage{ragged2e}

%%%%% AUTHORS - PLACE YOUR OWN MACROS HERE %%%%%
\usepackage{natbib}
\usepackage{amsfonts} 
\usepackage{amsbsy}
\usepackage{amsmath}
\usepackage[caption=false]{subfig}

\def\bse{\begin{eqnarray*}}
\def\ese{\end{eqnarray*}}
\def\be{\begin{eqnarray}}
\def\ee{\end{eqnarray}}
\newcommand{\ben}{\begin{eqnarray}}
\newcommand{\een}{\end{eqnarray}}
\newcommand{\bn}{\begin{enumerate}}
\newcommand{\en}{\end{enumerate}}
\newcommand{\im}{\item}

\def\bSig\pmb{\Sigma}
\newcommand{\VS}{V\&S}
\newcommand{\tr}{\mbox{tr}}
\def\wt{\widetilde}


%all vectors and matrix should be bolded in almost every statistical journal
\newcommand{\bda}{{\bf a}}
\newcommand{\bdb}{{\bf b}}
\newcommand{\bdc}{{\bf c}}
\newcommand{\bdd}{{\bf d}}
\newcommand{\bde}{{\bf e}}
\newcommand{\bdf}{{\bf f}}
\newcommand{\bdg}{{\bf g}}
\newcommand{\bdx}{{\bf x}}
\newcommand{\bdy}{{\bf y}}
\newcommand{\bdu}{{\bf u}}
\newcommand{\bdv}{{\bf v}}
\newcommand{\bdt}{{\bf t}}
\newcommand{\bds}{{\bf s}}
\newcommand{\bdq}{{\bf q}}
\newcommand{\bdone}{{\bf 1}}


\newcommand{\BZ}{{\pmb Z}}
\newcommand{\BX}{{\pmb X}}
\newcommand{\BY}{{\pmb Y}}
\newcommand{\BW}{{\pmb W}}
\newcommand{\BU}{{\pmb U}}
\newcommand{\Bb}{{\pmb b}}
\newcommand{\BI}{{\pmb I}}
\newcommand{\BJ}{{\pmb J}}
\newcommand{\BA}{{\pmb A}}
\newcommand{\BQ}{{\pmb Q}}
\newcommand{\BD}{{\pmb D}}
\newcommand{\BH}{{\pmb H}}
\newcommand{\BB}{{\pmb B}}
\newcommand{\BL}{{\pmb L}}
\newcommand{\BG}{{\pmb G}}
\newcommand{\BR}{{\pmb R}}
\newcommand{\Bone}{\pmb 1}
\newcommand{\Bbeta}{{\pmb \beta}}
\newcommand{\BSig}{{\pmb \Sigma}}
\newcommand{\Bdelta}{{\pmb \delta}}
\newcommand{\Bepsil}{{\pmb \epsilon}}
\newcommand{\Bpsi}{{\pmb \psi}}
\newcommand{\Bmu}{{\pmb \mu}}
\newcommand{\Bxi}{{\pmb \xi}}
\newcommand{\Bpi}{{\pmb \pi}}
\newcommand{\BPsi}{{\pmb \Psi}}
\newcommand{\Bgam}{{\pmb \gamma}}
\newcommand{\BGam}{{\pmb \Gamma}}
\newcommand{\Bal}{{\pmb \alpha}}
\newcommand{\BLam}{{\pmb \Lambda}}
\newcommand{\Cov}{\mathrm{Cov}}
\newcommand{\E}{\mathrm{E}}
\newcommand{\Var}{\mathrm{Var}}
\newcommand{\MZ}{{\mathbb Z}}
\newcommand{\MX}{{\mathbb X}}
\newcommand{\BV}{{\pmb V}}
\newcommand{\BThe}{{\pmb \Theta}}
\newcommand{\Bthe}{{\pmb \theta}}
\newcommand{\Beta}{{\pmb  \eta}}
\newcommand{\BOme}{{\pmb  \Omega}}
\newcommand{\MaThe}{\pmb {\Theta}}
\newcommand{\cov}{{\rm Cov}}

\newcommand{\CT}{{\cal T}}
\def\wt{\widetilde}
\def\diag{\hbox{diag}}
\def\wh{\widehat}
\def\wb{\overline}
\def\AIC{\hbox{AIC}}
\def\BIC{\hbox{BIC}}

%make it super bold for \pmb
\DeclareMathOperator{\Tr}{tr}

%  If you have a landscape table you need to use the rotating package
\usepackage{xcolor}
\newcommand{\yl}[1]{\textcolor{red}{#1}}
\newcommand\ylcm[1]{{\color{red}{[YL: #1]}}}

\usepackage[figuresright]{rotating}
\allowdisplaybreaks

%% \raggedbottom % To avoid glue in typesetteing, sbs>>

%%%%%%%%%%%%%%%%%%%%%%%%%%%%%%%%%%%%%%%%%%%%%%%%


\setcounter{footnote}{2}

\title[Semiparametric Mixture Regression for Asynchronous Longitudinal Data]{Codes for Semiparametric Mixture Regression for Asynchronous Longitudinal Data Using Multivariate Functional Principal Component Analysis}

\author{Ruihan Lu$^*$\email{rlu014@ucr.edu},
Yehua Li$^{**}$\email{yehua.li@ucr.edu}, and
Weixin Yao$^{***}$\email{weixin.yao@ucr.edu} \\
Department of Statistics, University of California, Riverside, CA, U.S.}



%----Helper code for dealing with external references----
% (by cyberSingularity at http://tex.stackexchange.com/a/69832/226)

\usepackage{xr}
\makeatletter
\newcommand*{\addFileDependency}[1]{% argument=file name and extension
  \typeout{(#1)}
  \@addtofilelist{#1}
  \IfFileExists{#1}{}{\typeout{No file #1.}}
}
\makeatother
\usepackage{lipsum}



\newcommand*{\myexternaldocument}[1]{%
    \externaldocument{#1}%
    \addFileDependency{#1.tex}%
    \addFileDependency{#1.aux}%
}

\def\blue{\color{blue}}
\def\red{\color{red}}
\def\yaocomment#1{\vskip 2mm\boxit{\vskip 2mm{\red #1} {\blue -- Yao\vskip 2mm}}\vskip 2mm}
%------------End of helper code--------------
\usepackage{titlesec}
% put all the external documents here!


\begin{document}

\maketitle

\begin{titlepage}
   \begin{center}
       \vspace*{1cm}

       \textbf{\Large{Package 'EMERALD'}}

       \vspace{0.5cm}
\begin{FlushLeft}
\textbf{Title} Semiparametric Mixture Regression for Asynchronous Longitudinal Data
\\
\textbf{Description} Estimation of asynchronous longitudinal data using multivariate functional principal component analysis; Estimation of the mixed models including latent class; Mixed models for multivariate longitudinal outcomes using a maximum likelihood estimation method

\end{FlushLeft}
            
            
   \end{center}
\end{titlepage}


%\tableofcontents

\section{SWAN folder}
\begin{itemize}
    \item basis1.csv file provides the nonorthogonal basis for $\BX_1$, which is glucose.
    \item  basis2.csv file provides the nonorthogonal basis for $\BX_2$, which is triglyceride.
    \item   basis3.csv file provides the nonorthogonal basis for $\BX_2$, which is systolic blood pressure.
    \item mu1.csv file provides the initial value for the mean function of glucose.
     \item mu2.csv file provides the initial value for the mean function of triglyceride.
      \item mu3.csv file provides the initial value for the mean function of systolic blood pressure.
      \item psi1.csv file provides the initial value for the eigenfunction of glucose;
      \item psi2.csv file provides the initial value for the eigenfunction of triglycerides.
      \item psi3.csv file provides the initial value for the eigenfunction of systolic blood pressure.
      \item probability1.csv, probability2.csv, probability3.csv, probability4.csv, and probability5.csv files provide the initial value of classification probability $(\pi_{ic})$ according to different number of clusters.
      \item new1class.py, new2class.py, new3class.py, new4class.py, and new5class.py files consist of functions \textbf{Emerald} need and return a text result file.
      \item W1.csv, W2.csv, W3.csv, and Ymeasure.csv are data files we would like to analyze.
      \item bootstrap for SWAN.py calculate the standard errors for each parameter using bootstrap data.
     \item result\textunderscore 1cluster\textunderscore hpcc.txt, result\textunderscore 2cluster\textunderscore hpcc.txt, result\textunderscore 3cluster\textunderscore hpcc.txt, result\textunderscore 4cluster\textunderscore hpcc.txt, and result\textunderscore 5cluster\textunderscore hpcc.txt files are the results from \textbf{Emerald}.
     \item After determining the number of subgroups, return the results as fit\textunderscore beta.csv, fit\textunderscore gamma.csv, fit\textunderscore mu.csv, fit\textunderscore psi.csv, and fit\textunderscore score.csv.
     \item parametric bootstap.R generates the bootstrap sample for SWAN data.
\end{itemize}



\section{Simulation study folder}
\subsection{2 subgroup folder}
\begin{itemize}
\item Corrected Simulation Data for 2 classes.ipynb simulates data for 2-subgroup.
    \item basis1.csv, basis2.csv provide the initial value of nonorthogonal basis for $\BX_1$, $\BX_2$, respectively.
\item mu1.csv, mu2.csv provide the initial value of the mean function for $\BX_1$, $\BX_2$, respectively.
\item psi1.csv, psi2.csv provide the initial value of the mean function for $\BX_1$, $\BX_2$, respectively.
\item probability2.csv provides the initial value of classification probability, which is $\pi_{ic}$.
\item result\textunderscore 2cluster\textunderscore hpcc.txt is the result file.
\end{itemize}

\subsection{3 subgroup}
\begin{itemize}
\item  Corrected Simulation Data for 3 classes.ipynb simulates data for 3-subgroup.
    \item basis1.csv, basis2.csv provide the initial value of nonorthogonal basis for $\BX_1$, $\BX_2$, respectively.
\item mu1.csv, mu2.csv provide the initial value of the mean function for $\BX_1$, $\BX_2$, respectively.
\item psi1.csv, psi2.csv provide the initial value of the mean function for $\BX_1$, $\BX_2$, respectively.
\item probability3.csv provides the initial value of classification probability, which is $\pi_{ic}$.
\item result\textunderscore 3cluster\textunderscore hpcc.txt is the result file.
\item  X1measure, X2measure, and Ymeasure folder contain datafile for 200 simulation runs.
\end{itemize}

\section{Detail function in \textit{Python} file}
\begin{itemize}
    \item fun\textunderscore df2dict (data): transfer each column into a nested dictionary.
    \item  fun\textunderscore diag\textunderscore blocks (data): transfer a $p \times 1$ vector into a $p \times p$ squared matrix.
    \item  fun\textunderscore b\textunderscore t\textunderscore N\textunderscore orth\textunderscore b\textunderscore t (n\textunderscore spline): construct a orthogonal matrix. The number of knots is required. 
    \item  fun\textunderscore theta\textunderscore mu (data\textunderscore basis, data\textunderscore original\textunderscore theta\textunderscore mu): calculate the matrix of observed mean function for variable $v$, which is $\BB_{iv} \Bthe_{\mu v}$.
    \item  fun\textunderscore theta\textunderscore psi (data\textunderscore orth\textunderscore basis, data\textunderscore original\textunderscore theta\textunderscore psi, data\textunderscore sigma, data\textunderscore pick\textunderscore n\textunderscore pc): after determining the number of principal components, calculate the matrix of observed eigenfunction for each variable $v$, which is $\BB_{iv} \BThe_{\psi v}$.
    \item fun\textunderscore matrix\textunderscore B\textunderscore iv (n, dx, data\textunderscore id, data\textunderscore W, data\textunderscore orth\textunderscore b\textunderscore t): calculate basis matrices for the observed time in all $\BW$ variables. Then return to the orthogonal basis. 
     \item fun\textunderscore matrix\textunderscore unorth\textunderscore B\textunderscore iv (n, dx, data\textunderscore id, data\textunderscore W, data\textunderscore orth\textunderscore b\textunderscore t): calculate basis matrices for the observed time in all $\BW$ variables. Then return to the nonorthogonal basis. 
     \item  fun\textunderscore matrix\textunderscore tilde\textunderscore mu\textunderscore iv (n, dx, data\textunderscore theta\textunderscore mu, data\textunderscore B\textunderscore unorth\textunderscore iv): given the nonorthogonal spline basis matrix, calculate the mean matrix for each subject within each variable of $\BW$.
     \item  fun\textunderscore matrix\textunderscore tilde\textunderscore psi\textunderscore iv(n, dx, data\textunderscore theta\textunderscore psi, data\textunderscore B\textunderscore iv): given the orthogonal spline basis matrix, calculate the eigenfunction matrix for each subject within each variable of $\BW$.
     \item fun\textunderscore matrix\textunderscore B\textunderscore star\textunderscore iv (n, dx, data\textunderscore id, data\textunderscore Y, data\textunderscore orth\textunderscore b\textunderscore t): calculate basis matrices for the observed time in $\BY$ variable. Then return to the orthogonal basis. 
     \item  fun\textunderscore matrix\textunderscore tilde\textunderscore mu\textunderscore star\textunderscore iv(n, dx, data\textunderscore theta\textunderscore mu, data\textunderscore B\textunderscore unorth\textunderscore star\textunderscore iv): given the nonorthogonal spline basis matrix, calculate the mean matrix for each subject within $\BY$.
     \item  fun\textunderscore matrix\textunderscore tilde\textunderscore mu\textunderscore star\textunderscore ic\textunderscore v(n, C, dx, data\textunderscore beta, data\textunderscore tilde\textunderscore mu\textunderscore star\textunderscore iv): multiple group-specific effect $\beta_c$ with mean matrix for each subject for all variables in $\BW$.
     \item fun\textunderscore matrix\textunderscore tilde\textunderscore psi\textunderscore star\textunderscore iv (n, dx, data\textunderscore theta\textunderscore psi, data\textunderscore B\textunderscore star\textunderscore iv): given the orthogonal spline basis matrix, calculate the eigenfunction matrix for each subject within $\BY$.
     \item  fun\textunderscore matrix\textunderscore tilde\textunderscore psi\textunderscore star\textunderscore ic\textunderscore v (n, C, dx, data\textunderscore beta, data\textunderscore tilde\textunderscore psi\textunderscore star\textunderscore iv): multiple group-specific effects $\beta_c$ with  a discrete matrix of eigenfunction for each subject for all variables in $\BW$.
     \item  fun\textunderscore matrix\textunderscore tilde\textunderscore W\textunderscore iv (n, dx, data\textunderscore id, data\textunderscore W, data\textunderscore tilde\textunderscore mu\textunderscore iv): calculate the value of $\wt{\BW}_{iv}$, which is $\BW_{iv} - \BB_{iv}^* \Bthe_{\mu v}$. 
     \item fun\textunderscore matrix\textunderscore tilde\textunderscore Y\textunderscore ic (n, C, dz, data\textunderscore id, data\textunderscore Z, data\textunderscore Y, data\textunderscore beta, data\textunderscore tilde\textunderscore mu\textunderscore star\textunderscore ic\textunderscore v): calculate the value of $\wt{\BY}_{ic}$, which is $\BY_{i} -\beta_{0,c} \Bone_{m_{y,i}} -\sum_{v=1}^{d_x} \beta_{x,cv} \BB_{iv} \Bthe_{\mu v}-\BZ_i \Bbeta_{z,c} $.
    \item fun\textunderscore matrix\textunderscore G\textunderscore y\textunderscore ic (n, C, data\textunderscore id, data\textunderscore Y, data\textunderscore var\textunderscore Y): construct variance-covariance matrix for each subject in the $c$th subgroup.
    \item  fun\textunderscore list\textunderscore invert\textunderscore lambda\textunderscore i (n, dx, data\textunderscore id, data\textunderscore W, data\textunderscore var\textunderscore W): construct the inverse matrix of variance in $\BW$ for each subject.
\end{itemize}






\end{document}
